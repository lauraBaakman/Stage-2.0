%!TEX root = paper.tex
% Introduction
\noindent In numerous fields kernel density estimation is a popular method to approximate probability densities.
% Method
Generally these methods use symmetric kernels, even though the data of which the density is estimated are not necessarily spread equally in all dimensions. To account for this asymmetric distribution of data we propose the use of shape adaptive kernels: kernels whose shape changes to fit the spread of the data in the local neighborhood.
% Experiment
We compare the performance of the shape adaptive kernels with that of an estimator that uses a symmetric kernel on simulated data sets with known density fields.
% Results
No significant differences in performance between the symmetric and the shape-adaptive estimator were found, although the former outperformed the latter on points near the boundary of the data sets. 
% Conclusion
In conclusion shape-adaptive kernels are a promising idea that warrants further research.
