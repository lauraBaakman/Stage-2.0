%!TEX root = paper.tex

%General Idea
	We use shape adaptive kernels in combination with the Modified Breiman Estimator introduced by \textcite{wilkinson1995dataplot} 
	%General bandwidth
	with the general bandwidths computed according to the method introduced by \textcite{ferdosi2011comparison} for its lower complexity.
	%Pilot Densities
	The pilot densities are computed on a $20 \times \cdots \times 20$ grid that covers all data points.
	%Local bandwidths
	We have empirically determined \todo{hoe hebben we dat vastgesteld} that using \varMBESensitivityParam = \todo{Een of andere waarde} works best in our case. 
	%Final densities
	The final densities are estimated according to \cref{eq:1:adaptiveKernelEstimateWithLocalBandwidths} with a reshaped and scaled Epanechnikov kernel. The shape of the kernel is determined by the bandwidth matrix \varBandwidthMatrix. Which is incorporated in the estimation of the densities by using 
	\begin{equation}\label{eq:2:parzenWithBandWidthMatrix}
		\varEstimatedDensityFunction{\varPattern} = \frac{1}{\varNumPatterns}\sum_{\itXis = 1}^{\varNumPatterns} \frac{1}{\det\left(\varBandwidthMatrix\right)}\varKernel{\varBandwidthMatrix^{-1}\left(\varPattern - \varPattern[\itXis]\right)}
	\end{equation}
	instead of \cref{eq:1:parzen} \cite{hardle2012nonparametric}. If $\varBandwidthMatrix = \varBandwidth \cdot \identityMatrix{\varDim}$ \cref{eq:2:parzenWithBandWidthMatrix} reduces to the fixed-width Parzen estimator defined in \cref{eq:1:parzen}. 
	
	For each data point \varPattern the bandwidth matrix is determined according to these steps:
		\begin{enumerate}[labelindent=0ex]
			\item \label{it:1:bandwidthmatrix:findNeighbours}
				Find \varNeighborhood{\varPattern}, the \KNNK-nearest neighbors of \varPattern.
			\item \label{it:1:bandwidthmatrix:initialBandwidthMatrix}
				Compute \varCovarianceMatrix, the biased covariance matrix  of local neighborhood \varNeighborhood{\varPattern}.
			\item \label{it:1:bandwidthmatrix:scaleBandwidhtMatrix}
				To determine \varBandwidthMatrix, scale \varCovarianceMatrix with
				\begin{equation}\label{eq:method:scalingFactor}
					\varScalingFactor = \varBandwidth^{\varDim} \cdot \left({\displaystyle\prod_{\itDim = 1}^{\varDim} \sqrt{\varEigenValue_\itDim}}\right)^{-1}.
				\end{equation}
		\end{enumerate}	
	% Discuss step 1
			% Why do we use KNN instead of fixed-radius
			Step \ref{it:1:bandwidthmatrix:findNeighbours} determines the local neighborhood of \varPattern with the \KNNK nearest neighbors algorithm (\KNN) with Euclidean distance. We use this approach rather than a fixed-radius neighborhood to ensure that, independent of the sparsity of the data, the kernel shape is always based on a reasonable number of data points. 
			% How do we choose K
			We follow \citeauthor{silverman1986density}'s \cite{silverman1986density} recommendation of choosing $k = \sqrt{\varNumPatterns}$. To ensure that \varCovarianceMatrix is nonsingular we also need $\KNNK \geq \varDim$, therefore
			\begin{equation*}
				\KNNK = \max\left(\left\lfloor \sqrt{\varNumPatterns} \right\rceil,\, \varDim \right).	
			\end{equation*}
		%  Discuss step 2
			Step \ref{it:1:bandwidthmatrix:initialBandwidthMatrix} determines the basic shape of the bandwidth covariance matrix. 

		% 	Discuss step 3
			Step \ref{it:1:bandwidthmatrix:scaleBandwidhtMatrix} ensures that the kernels used in the density estimation of different patterns have the same domain, modulo the local bandwidths. \Cref{eq:method:scalingFactor} scales the bandwidth matrix in such a way that the volume of the eigenellipse of \varBandwidthMatrix is equal to that of the eigenellipse of the bandwidth matrix that is implicitly used in \cref{eq:1:parzen}. 