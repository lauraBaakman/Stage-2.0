%!TEX root = paper.tex

%General Idea
	We use shape-adaptive kernels in combination with the Modified Breiman Estimator introduced by \textcite{wilkinson1995dataplot}. The resulting estimator is henceforth referred to as the shape-adaptive Modified Breiman Estimator (\sambe).
	%General bandwidth
	For its low computational complexity we use the method introduced by \citeauthor{ferdosi2011comparison}, defined in \cref{eq:1:ferdosiGeneralBandwidth}, to compute the general bandwidth.
	% Pilot Densities
	Pilot densities are computed according to the Parzen method, see \cref{eq:1:parzen}, with an Epanechnikov kernel.
	%Local bandwidths
	Since experiments indicated that using $\varMBESensitivityParam = \rfrac{1}{2}$ instead of $\varMBESensitivityParam = \rfrac{1}{\varDim}$ results in a final density approximation with a lower \MSE for most of our data sets, we use the first.
	%Final densities
	The final density estimate is computed according to:
	\begin{equation}\label{eq:2:parzenWithBandWidthMatrix}
		\varEstimatedDensityFunction{\varPattern} = \frac{1}{\varNumPatterns}\sum_{\itXis = 1}^{\varNumPatterns} \frac{1}{\det\left(\varBandwidthMatrix[\itXis]\right)}\varKernel[\varEpan]{\varBandwidthMatrix[\itXis]^{-1}\left(\varPattern - \varPattern[\itXis]\right)}.
	\end{equation}
	The shape of the kernel \varKernel[\varEpan]{\bullet} is determined by the bandwidth matrix \varBandwidthMatrix[\itXis] \cite{hardle2012nonparametric}. If $\varBandwidthMatrix[\itXis] = \varBandwidth \cdot \varLocalBandwidth{\itXis} \cdot \identityMatrix{\varDim}$, \cref{eq:2:parzenWithBandWidthMatrix} reduces to \cref{eq:1:adaptiveKernelEstimateWithLocalBandwidths}.

	For each data point \varPattern[\itXis] that is used in the density estimation of some pattern \varPattern[\itXs], the bandwidth matrix is determined according to these steps:
		\begin{enumerate}[labelindent=0ex]
			\item \label{it:1:bandwidthmatrix:findNeighbours}
				Find \varNeighborhood{\varPattern[\itXis]}, the \KNNK-nearest neighbors of \varPattern[\itXis].
			\item \label{it:1:bandwidthmatrix:initialBandwidthMatrix}
				Compute \varCovarianceMatrix, the unbiased covariance matrix of the local neighborhood \varNeighborhood{\varPattern[\itXis]}.
			\item \label{it:1:bandwidthmatrix:scaleBandwidhtMatrix}
				Determine \varBandwidthMatrix[\itXis] by multiplying \varCovarianceMatrix with a scaling factor:
				\begin{equation}\label{eq:method:scalingFactor}
					\varBandwidthMatrix[\itXis] = \varBandwidth \cdot \varLocalBandwidth{\itXis}
				 \left( \displaystyle\prod_{\itDim = 1}^{\varDim} \varEigenValue_\itDim \right)^{-\frac{1}{\varDim}} \cdot \varCovarianceMatrix
				\end{equation}
				where $\varEigenValue_1,\, \dotsc,\, \varEigenValue_\varDim$ are the eigenvalues of \varCovarianceMatrix.
		\end{enumerate}
	% Discuss step 1
		Step \ref{it:1:bandwidthmatrix:findNeighbours} determines the local neighborhood of \varPattern[\itXis]. We take Euclidean distance as the distance metric.
		% How do we choose K
		We follow \citeauthor{silverman1986density}'s \cite{silverman1986density} recommendation of choosing $k = \sqrt{\varNumPatterns}$. To ensure that \varCovarianceMatrix is nonsingular we also need $\KNNK > \varDim$, therefore
		\begin{equation}\label{eq:method:k}
			\KNNK = \max\left(\left\lfloor \sqrt{\varNumPatterns} \right\rceil,\, \varDim \right) + 1.
		\end{equation}
		% The KD-tree
		To reduce the time complexity of computing \varBandwidthMatrix[\itXis]we use a KD-tree for the \KNNK-nearest neighbors search\cite{Bentley1975Multidimensional}. The downside of our implementation of the space partitioning tree is that \varNeighborhood{\varPattern[\itXis]} is an approximation of the actual neighborhood. As long as \KNNK is rather large, the use of an approximation instead of the exact \KNNK-nearest neighbors should not impact the final kernel result too strongly.
		% Why do we use KNN instead of fixed-radius
		We use \KNN rather than a fixed-radius neighborhood to ensure that, independent of the sparsity of the data, the kernel shape is always based on a reasonable number of data points.

	%  Discuss step 2
		The basic shape of the kernel is determined in step \ref{it:1:bandwidthmatrix:initialBandwidthMatrix}. The covariance matrix ensures that the major axis of the kernel has the same direction as the maximum variance of the data.

	% 	Discuss step 3
		The scaling factor computed in step \ref{it:1:bandwidthmatrix:scaleBandwidhtMatrix} ensures that the kernels used in the density estimation of different patterns have a comparable domain. \Cref{eq:method:scalingFactor} scales the bandwidth matrix in such a way that the volume of the ellipsoid defined by the eigenvectors and values of \varBandwidthMatrix[\itXis] is equal to that of the eigenellipsoid of the bandwidth matrix that is implicitly used in \cref{eq:1:adaptiveKernelEstimateWithLocalBandwidths}.