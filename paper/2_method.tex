%!TEX root = paper.tex

%General Idea
	We use our shape adaptive kernels in combination with the Modified Breiman Estimator introduced by \textcite{wilkinson1995dataplot}. The grid used for the pilot densities is 
	%Pilot Densities
	The grid that the pilot densities are computed on \todo{Iets over hoe het grid bepaald wordt, of de standaard grootte van het grid.} 
	%General bandwidth
	We choose to use the method proposed by \textcite{ferdosi2011comparison} for computing the general bandwidth because of its lower complexity, compared to the method used by \textcite{wilkinson1995dataplot}. 
	%Local bandwidths
	We have empirically determined \todo{hoe hebben we dat vastgesteld} that using \varMBESensitivityParam = \todo{Een of andere waarde} works best in our case. 
	%Final densities
	The final densities are estimated according to \cref{eq:1:adaptiveKernelEstimateWithLocalBandwidths} with a reshaped and scaled Epanechnikov kernel. The Epanechnikov kernel reshaped with the matrix \varCovarianceMatrix is defined as:
	\begin{equation}\label{eq:1:epanechnikovKernelWithCovarianceMatrix}
		\varKernel[\varEpan]{\varPattern} = 
		\begin{cases}
			\frac{\varDim + 2}{2\varUnitSphere{\varDim}} \left( 1 - \varPattern \varCovarianceMatrix \varPattern \right) & \text{if } \varPattern \cdot \varPattern < 1\\
			0 & \text{otherwise.}
		\end{cases}
	\end{equation}
	As stated before the matrix \varCovarianceMatrix is determined based on the neighborhood of the pattern, \varPattern, whose density we are estimating. 

% The shape matrix
	We determine the neighbors of \varPattern with the \KNNK nearest neighbors algorithm (\KNN) with Euclidean distance. This approach is used rather than a fixed-radius neighborhood to ensure that independent of the sparsity of the data the kernel shape is always based on a reasonable number of data points. Furthermore using \KNN allows us to choose $\KNNK > \varDim$, which makes it extremely improbable that the covariance matrix of the neighborhood is singular. We follow \citeauthor{silverman1986density}'s \cite{silverman1986density} recommendation of choosing $k = \sqrt{\varNumPatterns}$. To ensure that even in high-dimensional data sets $\KNNK > \varDim$ we use
	\begin{equation*}
	\KNNK = \max\left(\sqrt{\varNumPatterns},\, \varDim + 1\right).	
	\end{equation*}
	Let \varNeighborhood{\varPattern} denote the union of \varPattern and its neighborhood, the basic shape of the kernel used for \varPattern is then given by \varCovarianceFunction{\varNeighborhood{\varPattern}}.

	\todo{The scaling discussed here has to change if the Epanechnikov kernel is used for the final density estimate.}
	\todo[inline]{For the uniformly scaled Gaussian this reduces to $h\sqrt{h}$.}
	To allow the density estimation of each pattern to be influenced by an equal area, before the application of the smoothing factor $\varLocalBandwidth{i}$, the basic shapes of the kernels need to be scaled. To that end we use the eigenellipse, the ellipse defined by the eigenvectors and eigenvalues of \varCovarianceFunction{\varNeighborhood{\varPattern}}. We scale the covariance matrix with the factor \varScalingFactor, defined as:
	\begin{equation}
		\varScalingFactor = \frac{\varBandwidth^2}{\varGeometricMeanFunction{\sqrt{\lambda_1}, \dotsc, \sqrt{\lambda_\varDim}}},
	\end{equation}
	where $\lambda_j$ denotes the $j$th eigenvalue of the $j$th eigenvector of \varCovarianceMatrix. The scaling factor \varScalingFactor ensures that the shape-adapted covariance matrix has the same scale as the covariance matrix that is implicitly used in the Modified Breiman Estimator with a Gaussian kernel.