%!TEX root = ../paper.tex
The simulated datasets are based on the simulated datasets used by \textcite{ferdosi2011comparison}. \Cref{fig:3:simulated:datasets} presents scatter plots of the data sets. The definitions of the these simulated data sets are shown in \cref{tab:3:simulated:datasets}.

\begin{figure*}
	\centering
	%!TEX root = ../paper.tex
%Ferdosi Sets 1/3
\begin{subfigure}{0.23\textwidth}
	\centering
	\includegraphics[width=\textwidth]{3/img/datasetplot_ferdosi_1_60000.pdf}
	\caption{Set \ferdosiOne}
	\label{fig:3:simulated:datasets:ferdosi1}
\end{subfigure}
\begin{subfigure}{0.23\textwidth}
	\centering
	\includegraphics[width=\textwidth]{3/img/datasetplot_ferdosi_2_60000.pdf}
	\caption{Set \ferdosiTwo}
	\label{fig:3:simulated:datasets:ferdosi2}
\end{subfigure}	
\begin{subfigure}{0.23\textwidth}
	\centering
	\includegraphics[width=\textwidth]{3/img/datasetplot_ferdosi_3_120000.pdf}
	\caption{Set \ferdosiThree}
	\label{fig:3:simulated:datasets:ferdosi3}
\end{subfigure}	
% Ferdosi Set 4
\begin{subfigure}{0.23\textwidth}
	\centering
	\includegraphics[width=\textwidth]{3/img/datasetplot_ferdosi_4_60000.pdf}
	\caption{Set \ferdosiFour}
	\label{fig:3:simulated:datasets:ferdosi4}
\end{subfigure}
% Baakman 1/3		
\begin{subfigure}{0.23\textwidth}
	\centering
	\includegraphics[width=\textwidth]{3/img/datasetplot_baakman_1_60000.pdf}
	\caption{Set \baakmanOne}
	\label{fig:3:simulated:datasets:baakman1}
\end{subfigure}
\begin{subfigure}{0.23\textwidth}
	\centering
	\includegraphics[width=\textwidth]{3/img/datasetplot_baakman_2_60000.pdf}
	\caption{Set \baakmanTwo}
	\label{fig:3:simulated:datasets:baakman2}
\end{subfigure}	
\begin{subfigure}{0.23\textwidth}
	\centering
	\includegraphics[width=\textwidth]{img/missingfigure.png}
	\caption{Set \baakmanThree}
	\label{fig:3:simulated:datasets:baakman3}
\end{subfigure}			
% Ferdosi 5
\begin{subfigure}{0.23\textwidth}
	\centering
	\includegraphics[width=\textwidth]{3/img/datasetplot_ferdosi_5_60000.pdf}
	\caption{Set \ferdosiFive}
	\label{fig:3:simulated:datasets:ferdosi5}
\end{subfigure}	
	\caption{Scatter plot representation of the simulated datasets defined in \cref{tab:3:simulated:datasets}. The components of the different datasets are coloured according to their order in \cref{tab:3:simulated:datasets}, the matching order of the colors is: red, blue, orange, green, yellow.}
	\label{fig:3:simulated:datasets}
\end{figure*}

\begin{table*}
	\centering
	%!TEX root = ../paper.tex

\begin{tabular}{@{}cclcl@{}}
\toprule
Set 		&~					& Component					& Fraction 				& Distribution\\
\midrule
% Ferdosi 1
\ferdosiOne 	&\legendDot{blue}	& Trivariate Gaussian 		& $\rfrac{2}{3}$		& $(x, y, z) \sim \gaussDist{[50, 50, 50]}{\diag(30)}$\\
~ 				&\legendDot{green}	& Uniform random background	& $\rfrac{1}{3}$		& $(x, y, z) \sim \uniformDist{[0, 0, 0]}{[100, 100, 100]}$\\
% Ferdosi 2
\hline
\ferdosiTwo 	&\legendDot{blue}	& Trivariate Gaussian 1		& $\rfrac{1}{3}$		& $(x, y, z) \sim \gaussDist{[25, 25, 25]}{\diag(5)}$\\
~ 				&\legendDot{green}	& Trivariate Gaussian 2		& $\rfrac{1}{3}$		& $(x, y, z) \sim \gaussDist{[65, 65, 65]}{\diag(20)}$\\
~ 				&\legendDot{red}	& Uniform random background	& $\rfrac{1}{3}$		& $(x, y, z) \sim \uniformDist{[0, 0, 0]}{[100, 100, 100]}$\\
% Ferdosi 3
\hline
\ferdosiThree	&\legendDot{blue}	& Trivariate Gaussian 1 	& $\rfrac{1}{6}$		& $(x, y, z) \sim \gaussDist{[24, 10, 10]}{\diag(2)}$\\
~ 				&\legendDot{green}	& Trivariate Gaussian 2 	& $\rfrac{1}{6}$		& $(x, y, z) \sim \gaussDist{[33, 70, 40]}{\diag(10)}$\\
~ 				&\legendDot{red}	& Trivariate Gaussian 3 	& $\rfrac{1}{6}$		& $(x, y, z) \sim \gaussDist{[90, 20, 80]}{\diag(1)}$\\
~ 				&\legendDot{orange}	& Trivariate Gaussian 4 	& $\rfrac{1}{6}$		& $(x, y, z) \sim \gaussDist{[60, 80, 23]}{\diag(5)}$\\
~ 				&\legendDot{purple}	& Uniform random background	& $\rfrac{1}{3}$		& $(x, y, z) \sim \uniformDist{[0, 0, 0]}{[100, 100, 100]}$\\
% Baakman 1
\hline
4 				&\legendDot{blue}	& Trivariate Gaussian 		& $\rfrac{2}{3}$		& $(x, y, z) \sim \gaussDist{[50, 50, 50]}{\diag{[9, \sqrt{3}} \sqrt{3}]}$\\
~ 				&\legendDot{green}	& Uniform random background	& $\rfrac{1}{3}$		& $(x, y, z) \sim \uniformDist{[0, 0, 0]}{[100, 100, 100]}$\\
% Baakman 2
\hline
5 				&\legendDot{blue}	& Trivariate Gaussian 1		& $\rfrac{1}{3}$		& $(x, y, z) \sim \gaussDist{[25, 25, 25]}{\diag(5)}$\\
~ 				&\legendDot{green}	& Trivariate Gaussian 2		& $\rfrac{1}{3}$		& $(x, y, z) \sim \gaussDist{[65, 65, 65]}{\diag(20)}$\\
~ 				&\legendDot{red}	& Uniform random background	& $\rfrac{1}{3}$		& $(x, y, z) \sim \uniformDist{[0, 0, 0]}{[100, 100, 100]}$\\
% Baakman 3
\hline
6 				&\legendDot{blue}	& Trivariate Gaussian 1 	& $\rfrac{1}{6}$		& $(x, y, z) \sim \gaussDist{[24, 10, 10]}{\diag(2)}$\\
~ 				&\legendDot{green}	& Trivariate Gaussian 2 	& $\rfrac{1}{6}$		& $(x, y, z) \sim \gaussDist{[33, 70, 40]}{\diag(10)}$\\
~ 				&\legendDot{red}	& Trivariate Gaussian 3 	& $\rfrac{1}{6}$		& $(x, y, z) \sim \gaussDist{[90, 20, 80]}{\diag(1)}$\\
~ 				&\legendDot{orange}	& Trivariate Gaussian 4 	& $\rfrac{1}{6}$		& $(x, y, z) \sim \gaussDist{[60, 80, 23]}{\diag(5)}$\\
~ 				&\legendDot{purple}	& Uniform random background	& $\rfrac{1}{3}$		& $(x, y, z) \sim \uniformDist{[0, 0, 0]}{[100, 100, 100]}$\\
% Ferdosi 4
\hline
\ferdosiFour 	&\legendDot{blue}	& Wall-like structure 		& $\rfrac{1}{2}$		& $(x, y) \sim \uniformDist{[0, 0]}{[100, 100]}$, $(z) \sim \gaussDist{50}{5}$\\
~ 				&\legendDot{green}	& Filament-like structure 	& $\rfrac{1}{2}$		& $(x, y) \sim \gaussDist{[50, 50]}{\diag(5)}$, $(z) \sim \uniformDist{0}{100}$\\
% Ferdosi 5
\hline
8 				&\legendDot{blue}	& Wall-like structure 1 	& $\rfrac{1}{3}$		& $(x, z) \sim \uniformDist{[0, 0]}{[100, 100]}$, $(y) \sim \gaussDist{10}{5}$\\
~ 				&\legendDot{green}	& Wall-like structure 2 	& $\rfrac{1}{3}$		& $(x, y) \sim \uniformDist{[0, 0]}{[100, 100]}$, $(z) \sim \gaussDist{50}{5}$\\
~ 				&\legendDot{red}	& Wall-like structure 3		& $\rfrac{1}{3}$		& $(x, z) \sim \uniformDist{[0, 0]}{[100, 100]}$, $(y) \sim \gaussDist{50}{5}$\\
\bottomrule
\end{tabular}
	\caption{The simulated datasets used to test the estimators. The column `Fraction' indicates for each component of the dataset which fraction of the total number of points of the data set is part of that component. \gaussDist{\varMean}{\varCovarianceMatrix} denotes a Gaussian distribution with mean \varMean and covariance matrix \varCovarianceMatrix. A diagonal matrix with the value $x$ on the diagonal is represented as $\diag(x)$. \uniformDist{a}{b} denotes a uniform distribution with its minimum and maximum set to $a$ and $b$, respectively.} 	
	\label{tab:3:simulated:datasets}
\end{table*}

\todo[inline]{Getjat van Ferdosi}
\todo[inline]{Verwijs naar tabel met de definities van de datasets en de plaatjes van de sets}

%Data Set 1
\todo[inline]{Dataset 1: Verwijs naar plaatje}
\todo[inline]{Dataset 1: Beschrijf dataset}
\todo[inline]{Dataset 1: Waarom is deze interessant}
\todo[inline]{Dataset 1: Wat verwachten we}

%Data Set 2
\todo[inline]{Dataset 2: Verwijs naar plaatje}
\todo[inline]{Dataset 2: Beschrijf dataset}
\todo[inline]{Dataset 2: Waarom is deze interessant}
\todo[inline]{Dataset 2: Wat verwachten we}	

%Data Set 3
\todo[inline]{Dataset 3: Verwijs naar plaatje}
\todo[inline]{Dataset 3: Beschrijf dataset}
\todo[inline]{Dataset 3: Waarom is deze interessant}
\todo[inline]{Dataset 3: Wat verwachten we}		

%Data Set 4
\todo[inline]{Dataset 4: Verwijs naar plaatje}
\todo[inline]{Dataset 4: Beschrijf dataset}
\todo[inline]{Dataset 4: Waarom is deze interessant}
\todo[inline]{Dataset 4: Wat verwachten we}			

%Data Set 5
\todo[inline]{Dataset 5: Verwijs naar plaatje}
\todo[inline]{Dataset 5: Beschrijf dataset}
\todo[inline]{Dataset 5: Waarom is deze interessant}
\todo[inline]{Dataset 5: Wat verwachten we}