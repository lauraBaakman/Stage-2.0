%!TEX root = ../paper.tex

The simulated datasets are a superset of the sets used by \textcite{ferdosi2011comparison}. \Cref{fig:3:simulated:datasets} shows scatter plots of these sets, their definition is given in \cref{tab:3:simulated:datasets}.

\begin{figure*}
	\centering
	\beforeFinalVersion{Remove ticks and labels.}
	%!TEX root = ../paper.tex
%Ferdosi Sets 1/3
\begin{subfigure}{0.23\textwidth}
	\centering
	\includegraphics[width=\textwidth]{3/img/datasetplot_ferdosi_1_60000.pdf}
	\caption{Set \ferdosiOne}
	\label{fig:3:simulated:datasets:ferdosi1}
\end{subfigure}
\begin{subfigure}{0.23\textwidth}
	\centering
	\includegraphics[width=\textwidth]{3/img/datasetplot_ferdosi_2_60000.pdf}
	\caption{Set \ferdosiTwo}
	\label{fig:3:simulated:datasets:ferdosi2}
\end{subfigure}	
\begin{subfigure}{0.23\textwidth}
	\centering
	\includegraphics[width=\textwidth]{3/img/datasetplot_ferdosi_3_120000.pdf}
	\caption{Set \ferdosiThree}
	\label{fig:3:simulated:datasets:ferdosi3}
\end{subfigure}	
% Ferdosi Set 4
\begin{subfigure}{0.23\textwidth}
	\centering
	\includegraphics[width=\textwidth]{3/img/datasetplot_ferdosi_4_60000.pdf}
	\caption{Set \ferdosiFour}
	\label{fig:3:simulated:datasets:ferdosi4}
\end{subfigure}
% Baakman 1/3		
\begin{subfigure}{0.23\textwidth}
	\centering
	\includegraphics[width=\textwidth]{3/img/datasetplot_baakman_1_60000.pdf}
	\caption{Set \baakmanOne}
	\label{fig:3:simulated:datasets:baakman1}
\end{subfigure}
\begin{subfigure}{0.23\textwidth}
	\centering
	\includegraphics[width=\textwidth]{3/img/datasetplot_baakman_2_60000.pdf}
	\caption{Set \baakmanTwo}
	\label{fig:3:simulated:datasets:baakman2}
\end{subfigure}	
\begin{subfigure}{0.23\textwidth}
	\centering
	\includegraphics[width=\textwidth]{img/missingfigure.png}
	\caption{Set \baakmanThree}
	\label{fig:3:simulated:datasets:baakman3}
\end{subfigure}			
% Ferdosi 5
\begin{subfigure}{0.23\textwidth}
	\centering
	\includegraphics[width=\textwidth]{3/img/datasetplot_ferdosi_5_60000.pdf}
	\caption{Set \ferdosiFive}
	\label{fig:3:simulated:datasets:ferdosi5}
\end{subfigure}				
	\caption{Scatter plot representation of the simulated datasets defined in \cref{tab:3:simulated:datasets}. }
	\label{fig:3:simulated:datasets}
\end{figure*}

\begin{table*}
	\centering
	%!TEX root = ../paper.tex

\begin{tabular}{@{}cclcl@{}}
\toprule
Set 		&~					& Component					& Fraction 				& Distribution\\
\midrule
% Ferdosi 1
\ferdosiOne 	&\legendDot{blue}	& Trivariate Gaussian 		& $\rfrac{2}{3}$		& $(x, y, z) \sim \gaussDist{[50, 50, 50]}{\diag(30)}$\\
~ 				&\legendDot{green}	& Uniform random background	& $\rfrac{1}{3}$		& $(x, y, z) \sim \uniformDist{[0, 0, 0]}{[100, 100, 100]}$\\
% Ferdosi 2
\hline
\ferdosiTwo 	&\legendDot{blue}	& Trivariate Gaussian 1		& $\rfrac{1}{3}$		& $(x, y, z) \sim \gaussDist{[25, 25, 25]}{\diag(5)}$\\
~ 				&\legendDot{green}	& Trivariate Gaussian 2		& $\rfrac{1}{3}$		& $(x, y, z) \sim \gaussDist{[65, 65, 65]}{\diag(20)}$\\
~ 				&\legendDot{red}	& Uniform random background	& $\rfrac{1}{3}$		& $(x, y, z) \sim \uniformDist{[0, 0, 0]}{[100, 100, 100]}$\\
% Ferdosi 3
\hline
\ferdosiThree	&\legendDot{blue}	& Trivariate Gaussian 1 	& $\rfrac{1}{6}$		& $(x, y, z) \sim \gaussDist{[24, 10, 10]}{\diag(2)}$\\
~ 				&\legendDot{green}	& Trivariate Gaussian 2 	& $\rfrac{1}{6}$		& $(x, y, z) \sim \gaussDist{[33, 70, 40]}{\diag(10)}$\\
~ 				&\legendDot{red}	& Trivariate Gaussian 3 	& $\rfrac{1}{6}$		& $(x, y, z) \sim \gaussDist{[90, 20, 80]}{\diag(1)}$\\
~ 				&\legendDot{orange}	& Trivariate Gaussian 4 	& $\rfrac{1}{6}$		& $(x, y, z) \sim \gaussDist{[60, 80, 23]}{\diag(5)}$\\
~ 				&\legendDot{purple}	& Uniform random background	& $\rfrac{1}{3}$		& $(x, y, z) \sim \uniformDist{[0, 0, 0]}{[100, 100, 100]}$\\
% Baakman 1
\hline
4 				&\legendDot{blue}	& Trivariate Gaussian 		& $\rfrac{2}{3}$		& $(x, y, z) \sim \gaussDist{[50, 50, 50]}{\diag{[9, \sqrt{3}} \sqrt{3}]}$\\
~ 				&\legendDot{green}	& Uniform random background	& $\rfrac{1}{3}$		& $(x, y, z) \sim \uniformDist{[0, 0, 0]}{[100, 100, 100]}$\\
% Baakman 2
\hline
5 				&\legendDot{blue}	& Trivariate Gaussian 1		& $\rfrac{1}{3}$		& $(x, y, z) \sim \gaussDist{[25, 25, 25]}{\diag(5)}$\\
~ 				&\legendDot{green}	& Trivariate Gaussian 2		& $\rfrac{1}{3}$		& $(x, y, z) \sim \gaussDist{[65, 65, 65]}{\diag(20)}$\\
~ 				&\legendDot{red}	& Uniform random background	& $\rfrac{1}{3}$		& $(x, y, z) \sim \uniformDist{[0, 0, 0]}{[100, 100, 100]}$\\
% Baakman 3
\hline
6 				&\legendDot{blue}	& Trivariate Gaussian 1 	& $\rfrac{1}{6}$		& $(x, y, z) \sim \gaussDist{[24, 10, 10]}{\diag(2)}$\\
~ 				&\legendDot{green}	& Trivariate Gaussian 2 	& $\rfrac{1}{6}$		& $(x, y, z) \sim \gaussDist{[33, 70, 40]}{\diag(10)}$\\
~ 				&\legendDot{red}	& Trivariate Gaussian 3 	& $\rfrac{1}{6}$		& $(x, y, z) \sim \gaussDist{[90, 20, 80]}{\diag(1)}$\\
~ 				&\legendDot{orange}	& Trivariate Gaussian 4 	& $\rfrac{1}{6}$		& $(x, y, z) \sim \gaussDist{[60, 80, 23]}{\diag(5)}$\\
~ 				&\legendDot{purple}	& Uniform random background	& $\rfrac{1}{3}$		& $(x, y, z) \sim \uniformDist{[0, 0, 0]}{[100, 100, 100]}$\\
% Ferdosi 4
\hline
\ferdosiFour 	&\legendDot{blue}	& Wall-like structure 		& $\rfrac{1}{2}$		& $(x, y) \sim \uniformDist{[0, 0]}{[100, 100]}$, $(z) \sim \gaussDist{50}{5}$\\
~ 				&\legendDot{green}	& Filament-like structure 	& $\rfrac{1}{2}$		& $(x, y) \sim \gaussDist{[50, 50]}{\diag(5)}$, $(z) \sim \uniformDist{0}{100}$\\
% Ferdosi 5
\hline
8 				&\legendDot{blue}	& Wall-like structure 1 	& $\rfrac{1}{3}$		& $(x, z) \sim \uniformDist{[0, 0]}{[100, 100]}$, $(y) \sim \gaussDist{10}{5}$\\
~ 				&\legendDot{green}	& Wall-like structure 2 	& $\rfrac{1}{3}$		& $(x, y) \sim \uniformDist{[0, 0]}{[100, 100]}$, $(z) \sim \gaussDist{50}{5}$\\
~ 				&\legendDot{red}	& Wall-like structure 3		& $\rfrac{1}{3}$		& $(x, z) \sim \uniformDist{[0, 0]}{[100, 100]}$, $(y) \sim \gaussDist{50}{5}$\\
\bottomrule
\end{tabular}
	\caption{The simulated datasets used to test the estimators. The column `Fraction' indicates for each component of the dataset which fraction of the total number of points of the data set is part of that component. \gaussDist{\varMean}{\varCovarianceMatrix} denotes a Gaussian distribution with mean \varMean and covariance matrix \varCovarianceMatrix. A diagonal matrix with the value $x$ on the diagonal is represented as $\diag(x)$. \uniformDist{a}{b} denotes a uniform distribution with its minimum and maximum set to $a$ and $b$, respectively. The colors shown in the second column correspond with the colors used for these components of the data set throughout the paper.} 	
	\label{tab:3:simulated:datasets}
\end{table*}

%Data Set Ferdosi 1
	Dataset \ferdosiOne, shown in \cref{fig:3:simulated:datasets:ferdosi1}, is the most simple set in this group. It is an unimodal Gaussian distribution with random noise added. 

%Data Set Ferdosi 2
	The \ordinalstringnum{\ferdosiTwoNum} dataset, depicted in \cref{fig:3:simulated:datasets:ferdosi2}, contains two Gaussian distributions with different covariance matrices and uniform noise. The means and covariance matrices of the Normal distributions are such that the two distributions are very unlikely to overlap. 

%Data Set Ferdosi 3
	Dataset \ferdosiThree, represented in \cref{fig:3:simulated:datasets:ferdosi3}, consists of four different normal distributions and uniformly distributed noise. The four Gaussian distributions are placed in such a way that it is unlikely that among them any overlap occurs.

%Data Set Baakman 1
	\todo[inline]{Baakman 1}

%Data Set Baakman 2
	\todo[inline]{Baakman 2}

%Data Set Baakman 3
	\todo[inline]{Baakman 3}

%Data Set Ferdosi 4
	\Cref{fig:3:simulated:datasets:ferdosi4} illustrates dataset \ferdosiFour. This set consists of a horizontal wall-like structure and a vertical filament-like structure. 

%Data Set Ferdosi 5
	The \ordinalstringnum{\ferdosiFiveNum} dataset, shown in \cref{fig:3:simulated:datasets:ferdosi5}, contains three intersecting walls. For each point in these walls its position in two of the three dimensions is drawn from a uniform distribution, the third coordinate is sampled from a Gaussian distribution.


%Expectations for the datasets
We expect comparable performance from all estimators on dataset one through three, as other than the randomly sampled noise these sets only contain data sampled from a Gaussian distribution with a diagonal covariance matrix. Which results in an equal spread of the data in all dimensions for the non-noise data. Dataset four and five are clearly spread more in one dimension than in others, thus we expect the shape adaptive estimator to outperform the MBE estimator.

%Conclusion
The increasing complexity of these datasets allows us to investigate the performance of the classifier on simple situations, one cluster of data with some noise, to complex density fields that should better approximate real world data. The advantage of using simulated data is that the true densities of the data points are known, which allows us to test how well the different methods estimate the densities.