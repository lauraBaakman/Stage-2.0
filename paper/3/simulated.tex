%!TEX root = ../paper.tex

The simulated datasets are a superset of the sets used by \textcite{ferdosi2011comparison}. \Cref{fig:3:simulated:datasets} shows scatter plots of these sets, their definition is given in \cref{tab:3:simulated:datasets}.

\begin{figure*}
	\centering
	\beforeFinalVersion{Remove ticks and labels.}
	%!TEX root = ../paper.tex
%Ferdosi Sets 1/3
\begin{subfigure}{0.23\textwidth}
	\centering
	\includegraphics[width=\textwidth]{3/img/datasetplot_ferdosi_1_60000.pdf}
	\caption{Set \ferdosiOne}
	\label{fig:3:simulated:datasets:ferdosi1}
\end{subfigure}
\begin{subfigure}{0.23\textwidth}
	\centering
	\includegraphics[width=\textwidth]{3/img/datasetplot_ferdosi_2_60000.pdf}
	\caption{Set \ferdosiTwo}
	\label{fig:3:simulated:datasets:ferdosi2}
\end{subfigure}	
\begin{subfigure}{0.23\textwidth}
	\centering
	\includegraphics[width=\textwidth]{3/img/datasetplot_ferdosi_3_120000.pdf}
	\caption{Set \ferdosiThree}
	\label{fig:3:simulated:datasets:ferdosi3}
\end{subfigure}	
% Ferdosi Set 4
\begin{subfigure}{0.23\textwidth}
	\centering
	\includegraphics[width=\textwidth]{3/img/datasetplot_ferdosi_4_60000.pdf}
	\caption{Set \ferdosiFour}
	\label{fig:3:simulated:datasets:ferdosi4}
\end{subfigure}
% Baakman 1/3		
\begin{subfigure}{0.23\textwidth}
	\centering
	\includegraphics[width=\textwidth]{3/img/datasetplot_baakman_1_60000.pdf}
	\caption{Set \baakmanOne}
	\label{fig:3:simulated:datasets:baakman1}
\end{subfigure}
\begin{subfigure}{0.23\textwidth}
	\centering
	\includegraphics[width=\textwidth]{3/img/datasetplot_baakman_2_60000.pdf}
	\caption{Set \baakmanTwo}
	\label{fig:3:simulated:datasets:baakman2}
\end{subfigure}	
\begin{subfigure}{0.23\textwidth}
	\centering
	\includegraphics[width=\textwidth]{img/missingfigure.png}
	\caption{Set \baakmanThree}
	\label{fig:3:simulated:datasets:baakman3}
\end{subfigure}			
% Ferdosi 5
\begin{subfigure}{0.23\textwidth}
	\centering
	\includegraphics[width=\textwidth]{3/img/datasetplot_ferdosi_5_60000.pdf}
	\caption{Set \ferdosiFive}
	\label{fig:3:simulated:datasets:ferdosi5}
\end{subfigure}				
	\todo[inline]{Update caption with warning about the order.}
	\caption{Scatter plot representation of the simulated datasets defined in \cref{tab:3:simulated:datasets}. }
	\label{fig:3:simulated:datasets}
\end{figure*}

\begin{table*}
	\centering
	%!TEX root = ../paper.tex

\begin{tabular}{@{}cclcl@{}}
\toprule
Set 		&~					& Component					& Fraction 				& Distribution\\
\midrule
% Ferdosi 1
\ferdosiOne 	&\legendDot{blue}	& Trivariate Gaussian 		& $\rfrac{2}{3}$		& $(x, y, z) \sim \gaussDist{[50, 50, 50]}{\diag(30)}$\\
~ 				&\legendDot{green}	& Uniform random background	& $\rfrac{1}{3}$		& $(x, y, z) \sim \uniformDist{[0, 0, 0]}{[100, 100, 100]}$\\
% Ferdosi 2
\hline
\ferdosiTwo 	&\legendDot{blue}	& Trivariate Gaussian 1		& $\rfrac{1}{3}$		& $(x, y, z) \sim \gaussDist{[25, 25, 25]}{\diag(5)}$\\
~ 				&\legendDot{green}	& Trivariate Gaussian 2		& $\rfrac{1}{3}$		& $(x, y, z) \sim \gaussDist{[65, 65, 65]}{\diag(20)}$\\
~ 				&\legendDot{red}	& Uniform random background	& $\rfrac{1}{3}$		& $(x, y, z) \sim \uniformDist{[0, 0, 0]}{[100, 100, 100]}$\\
% Ferdosi 3
\hline
\ferdosiThree	&\legendDot{blue}	& Trivariate Gaussian 1 	& $\rfrac{1}{6}$		& $(x, y, z) \sim \gaussDist{[24, 10, 10]}{\diag(2)}$\\
~ 				&\legendDot{green}	& Trivariate Gaussian 2 	& $\rfrac{1}{6}$		& $(x, y, z) \sim \gaussDist{[33, 70, 40]}{\diag(10)}$\\
~ 				&\legendDot{red}	& Trivariate Gaussian 3 	& $\rfrac{1}{6}$		& $(x, y, z) \sim \gaussDist{[90, 20, 80]}{\diag(1)}$\\
~ 				&\legendDot{orange}	& Trivariate Gaussian 4 	& $\rfrac{1}{6}$		& $(x, y, z) \sim \gaussDist{[60, 80, 23]}{\diag(5)}$\\
~ 				&\legendDot{purple}	& Uniform random background	& $\rfrac{1}{3}$		& $(x, y, z) \sim \uniformDist{[0, 0, 0]}{[100, 100, 100]}$\\
% Baakman 1
\hline
4 				&\legendDot{blue}	& Trivariate Gaussian 		& $\rfrac{2}{3}$		& $(x, y, z) \sim \gaussDist{[50, 50, 50]}{\diag{[9, \sqrt{3}} \sqrt{3}]}$\\
~ 				&\legendDot{green}	& Uniform random background	& $\rfrac{1}{3}$		& $(x, y, z) \sim \uniformDist{[0, 0, 0]}{[100, 100, 100]}$\\
% Baakman 2
\hline
5 				&\legendDot{blue}	& Trivariate Gaussian 1		& $\rfrac{1}{3}$		& $(x, y, z) \sim \gaussDist{[25, 25, 25]}{\diag(5)}$\\
~ 				&\legendDot{green}	& Trivariate Gaussian 2		& $\rfrac{1}{3}$		& $(x, y, z) \sim \gaussDist{[65, 65, 65]}{\diag(20)}$\\
~ 				&\legendDot{red}	& Uniform random background	& $\rfrac{1}{3}$		& $(x, y, z) \sim \uniformDist{[0, 0, 0]}{[100, 100, 100]}$\\
% Baakman 3
\hline
6 				&\legendDot{blue}	& Trivariate Gaussian 1 	& $\rfrac{1}{6}$		& $(x, y, z) \sim \gaussDist{[24, 10, 10]}{\diag(2)}$\\
~ 				&\legendDot{green}	& Trivariate Gaussian 2 	& $\rfrac{1}{6}$		& $(x, y, z) \sim \gaussDist{[33, 70, 40]}{\diag(10)}$\\
~ 				&\legendDot{red}	& Trivariate Gaussian 3 	& $\rfrac{1}{6}$		& $(x, y, z) \sim \gaussDist{[90, 20, 80]}{\diag(1)}$\\
~ 				&\legendDot{orange}	& Trivariate Gaussian 4 	& $\rfrac{1}{6}$		& $(x, y, z) \sim \gaussDist{[60, 80, 23]}{\diag(5)}$\\
~ 				&\legendDot{purple}	& Uniform random background	& $\rfrac{1}{3}$		& $(x, y, z) \sim \uniformDist{[0, 0, 0]}{[100, 100, 100]}$\\
% Ferdosi 4
\hline
\ferdosiFour 	&\legendDot{blue}	& Wall-like structure 		& $\rfrac{1}{2}$		& $(x, y) \sim \uniformDist{[0, 0]}{[100, 100]}$, $(z) \sim \gaussDist{50}{5}$\\
~ 				&\legendDot{green}	& Filament-like structure 	& $\rfrac{1}{2}$		& $(x, y) \sim \gaussDist{[50, 50]}{\diag(5)}$, $(z) \sim \uniformDist{0}{100}$\\
% Ferdosi 5
\hline
8 				&\legendDot{blue}	& Wall-like structure 1 	& $\rfrac{1}{3}$		& $(x, z) \sim \uniformDist{[0, 0]}{[100, 100]}$, $(y) \sim \gaussDist{10}{5}$\\
~ 				&\legendDot{green}	& Wall-like structure 2 	& $\rfrac{1}{3}$		& $(x, y) \sim \uniformDist{[0, 0]}{[100, 100]}$, $(z) \sim \gaussDist{50}{5}$\\
~ 				&\legendDot{red}	& Wall-like structure 3		& $\rfrac{1}{3}$		& $(x, z) \sim \uniformDist{[0, 0]}{[100, 100]}$, $(y) \sim \gaussDist{50}{5}$\\
\bottomrule
\end{tabular}
	\todo[inline]{Remove fraction, show number of patterns, use siunitx number capabilities to show exponential numbers}
	\caption{The simulated datasets used to test the estimators. The column `Fraction' indicates for each component of the dataset which fraction of the total number of points of the data set is part of that component. \gaussDist{\varMean}{\varCovarianceMatrix} denotes a Gaussian distribution with mean \varMean and covariance matrix \varCovarianceMatrix. A diagonal matrix with the values $x_1,\, \cdots,\, x_\varDim$ on the diagonal is represented as $\diag([x_1,\,\cdots,\,x_\varDim]])$, a scalar matrix with $x$ on the diagonal is shown as $\diag(x)$. \uniformDist{a}{b} denotes a uniform distribution with its minimum and maximum set to $a$ and $b$, respectively. The colors shown in the second column correspond with the colors used for these components of the data set throughout the paper.} 	
	\label{tab:3:simulated:datasets}
\end{table*}

% Ferdosi 1/3
Dataset \numberstringnum{\ferdosiOneNum} through \numberstringnum{\ferdosiThreeNum}, shown in \crefrange{fig:3:simulated:datasets:ferdosi1}{fig:3:simulated:datasets:ferdosi3}, are taken from \textcite{ferdosi2011comparison}. They consist of a number of spherical Gaussian distributions with random noise added. The means of the Gaussian distribution are chosen in such a way that it is unlikely that the distributions overlap.

%Baakman 1/3
\Crefrange{fig:3:simulated:datasets:baakman1}{fig:3:simulated:datasets:baakman3} present dataset \numberstringnum{\baakmanOneNum} through \numberstringnum{\baakmanThreeNum}. These datasets are created from \numberstringnum{\ferdosiOneNum} through \numberstringnum{\ferdosiThreeNum} in such a way that the volumes of the eigenellipsoids defined by the covariance matrices of the components of these datasets are equal to volumes of the eigenspheres of the covariance matrix of the component in dataset \numberstringnum{\ferdosiOneNum} through \numberstringnum{\ferdosiThreeNum} that inspired them. Furthermore if $a$ is the eigenvalue of the original covariance matrix, the eigenvalues of the covariance matrix in dataset \numberstringnum{\baakmanOneNum} through \numberstringnum{\baakmanThreeNum} are $a^2$, $\sqrt{a}$ and $\sqrt{a}$. Consequently the minor axes of these datasets have the same length.

% Baakman 4/5
In dataset \numberstringnum{\baakmanFourNum} and \numberstringnum{\baakmanFiveNum}, illustrated in \crefrange{fig:3:simulated:datasets:baakman3}{fig:3:simulated:datasets:baakman4} the semi axes of the ellipsoids all have a different length. The largest minor axis of the Trivariate Gaussian in dataset \numberstringnum{\baakmanFourNum} is a factor two larger than the smallest minor axis in that dataset. Whereas in dataset \numberstringnum{\baakmanFiveNum} the largest minor axis is exponentially larger than the smallest minor axis. 

%Data Set Ferdosi 4
	\Cref{fig:3:simulated:datasets:ferdosi4} illustrates dataset \numberstringnum{\ferdosiFourNum}. This set consists of a horizontal wall-like structure and a vertical filament-like structure. 

%Data Set Ferdosi 5
	The \ordinalstringnum{\ferdosiFiveNum} dataset, shown in \cref{fig:3:simulated:datasets:ferdosi5}, contains three intersecting walls. For each point in these walls its position in two of the three dimensions is drawn from a uniform distribution, the third coordinate is sampled from a Gaussian distribution.

%Expectations for the datasets
% Ferdosi 1/3
We expect comparable performance from all estimators on dataset \numberstringnum{\ferdosiOneNum} through \numberstringnum{\ferdosiThreeNum}, as other than the randomly sampled noise these sets only contain data sampled from a Gaussian distribution with a diagonal covariance matrix. Which results in an equal spread of the data in all dimensions for the non-noise data. 
% Baakman 1/5
Given the elongated shape of the non-noise components in \numberstringnum{\baakmanOneNum} through \numberstringnum{\baakmanFiveNum} we expect the shape-adaptive estimator to outperform the MBE. 
% Ferdosi 4/5
Dataset \numberstringnum{\ferdosiFourNum} and \numberstringnum{\ferdosiFiveNum} are clearly spread more in one dimension than in other dimensions, thus we expect the shape adaptive estimator to outperform the MBE estimator.

%Conclusion
The increasing complexity of these datasets allows us to investigate the performance of the classifier on simple situations, one cluster of data with some noise, to complex density fields that approximate real world data. The advantage of using simulated data is that the true densities of the data points are known, which allows us to test how well the different methods estimate the densities.