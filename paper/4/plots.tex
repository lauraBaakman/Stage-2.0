%!TEX root = ../paper.tex
This section presents a visual representation of the results of the estimators that allows us to compare the performance of the two estimators on a single dataset. All plots associated with a single dataset have the same axes. The horizontal axis is used to represent the known densities, this axis has the same range as the known densities. The estimated densities are shown on the vertical axis, the length of these axes is such that they are long enough to represent every estimated density for that dataset, independent of the used estimator. The black line in each plot illustrates the line all points would lie on if a perfect estimator was used, \ie the line $x = x$. 

\begin{figure*}
	\centering
	%!TEX root = ../paper.tex

% Ferdosi 1
\begin{subfigure}{0.23\textwidth}
	\centering
	\includegraphics[keepaspectratio=true, width=\textwidth, height=0.23\textheight]{4/img/all/results_ferdosi_1_60000_mbe_silverman}
	\caption{Set \ferdosiOne, \mbe}
	\label{fig:4:results:mbe:ferdosi1}
\end{subfigure}
\begin{subfigure}{0.23\textwidth}
	\centering
	\includegraphics[keepaspectratio=true, width=\textwidth, height=0.23\textheight]{4/img/all/results_ferdosi_1_60000_sambe_silverman}
	\caption{Set \ferdosiOne, \sambe}
	\label{fig:4:results:sambe:ferdosi1}
\end{subfigure}
% Baakman 1	
\begin{subfigure}{0.23\textwidth}
	\centering
	\includegraphics[keepaspectratio=true, width=\textwidth, height=0.23\textheight]{4/img/all/results_baakman_1_60000_mbe_silverman}
	\caption{Set \baakmanOne, \mbe}
	\label{fig:4:results:mbe:baakman1}
\end{subfigure}
\begin{subfigure}{0.23\textwidth}
	\centering
	\includegraphics[keepaspectratio=true, width=\textwidth, height=0.23\textheight]{4/img/all/results_baakman_1_60000_sambe_silverman}
	\caption{Set \baakmanOne, \sambe}
	\label{fig:4:results:sambe:baakman1}
\end{subfigure}
% Baakman 4
\begin{subfigure}{0.23\textwidth}
	\centering
	\includegraphics[keepaspectratio=true, width=\textwidth, height=0.23\textheight]{4/img/all/results_baakman_4_60000_mbe_silverman}
	\caption{Set \baakmanFour, \mbe}
	\label{fig:4:results:mbe:baakman4}
\end{subfigure}	
\begin{subfigure}{0.23\textwidth}
	\centering
	\includegraphics[keepaspectratio=true, width=\textwidth, height=0.23\textheight]{4/img/all/results_baakman_4_60000_sambe_silverman}
	\caption{Set \baakmanFour, \sambe}
	\label{fig:4:results:sambe:baakman4}
\end{subfigure}	
% Baakman 5
\begin{subfigure}{0.23\textwidth}
	\centering
	\includegraphics[keepaspectratio=true, width=\textwidth, height=0.23\textheight]{4/img/all/results_baakman_5_60000_mbe_silverman}
	\caption{Set \baakmanFive, \mbe}
	\label{fig:4:results:mbe:baakman5}
\end{subfigure}	
\begin{subfigure}{0.23\textwidth}
	\centering
	\includegraphics[keepaspectratio=true, width=\textwidth, height=0.23\textheight]{4/img/all/results_baakman_5_60000_sambe_silverman}
	\caption{Set \baakmanFive, \sambe}
	\label{fig:4:results:sambe:baakman5}
\end{subfigure}	
	\caption{Comparative plots for dataset \ferdosiOne, \baakmanOne, \baakmanFour, \baakmanFive.}
	\label{fig:4:results:singleSphere}
\end{figure*}

\todo[inline]{What do we observe for dataset ferdosi 1, baakman 1, baakman 4, baakman 5}


\begin{figure*}
	\centering
	%!TEX root = ../paper.tex
\todo[inline]{Point to \cref{fig:4:resuts:multiSphere}}.

\begin{figure*}
	\centering
	%!TEX root = ../paper.tex

% Ferdosi 2
\begin{subfigure}{0.23\textwidth}
	\centering
	\includegraphics[keepaspectratio=true, width=\textwidth, height=0.23\textheight]{result/img/all/results_ferdosi_2_60000_mbe_silverman}
	\caption{Set \ferdosiTwo, \mbe}
	\label{fig:4:results:mbe:ferdosi2}
\end{subfigure}
\begin{subfigure}{0.23\textwidth}
	\centering
	\includegraphics[keepaspectratio=true, width=\textwidth, height=0.23\textheight]{result/img/all/results_ferdosi_2_60000_sambe_silverman}
	\caption{Set \ferdosiTwo, \sambe}
	\label{fig:4:results:sambe:ferdosi2}
\end{subfigure}
% Baakman 2
\begin{subfigure}{0.23\textwidth}
	\centering
	\includegraphics[keepaspectratio=true, width=\textwidth, height=0.23\textheight]{result/img/all/results_baakman_2_60000_mbe_silverman}
	\caption{Set \baakmanTwo, \mbe}
	\label{fig:4:results:mbe:baakman2}
\end{subfigure}
\begin{subfigure}{0.23\textwidth}
	\centering
	\includegraphics[keepaspectratio=true, width=\textwidth, height=0.23\textheight]{result/img/all/results_baakman_2_60000_sambe_silverman}
	\caption{Set \baakmanTwo, \sambe}
	\label{fig:4:simulated:datasets:sambe:baakman2}
\end{subfigure}
% Ferdosi 3
\begin{subfigure}{0.23\textwidth}
	\centering
	\includegraphics[keepaspectratio=true, width=\textwidth, height=0.23\textheight]{result/img/all/results_ferdosi_3_120000_mbe_silverman.png}
	\caption{Set \ferdosiThree, \mbe}
	\label{fig:4:results:mbe:ferdosi3}
\end{subfigure}
\begin{subfigure}{0.23\textwidth}
	\centering
	\includegraphics[keepaspectratio=true, width=\textwidth, height=0.23\textheight]{result/img/all/results_ferdosi_3_120000_sambe_silverman}
	\caption{Set \ferdosiThree, \sambe}
	\label{fig:4:simulated:datasets:sambe:ferdosi3}
\end{subfigure}
% Baakman 3
\begin{subfigure}{0.23\textwidth}
	\centering
	\includegraphics[keepaspectratio=true, width=\textwidth, height=0.23\textheight]{result/img/all/results_baakman_3_120000_mbe_silverman}
	\caption{Set \baakmanThree, \mbe}
	\label{fig:4:results:mbe:baakman3}
\end{subfigure}	
\begin{subfigure}{0.23\textwidth}
	\centering
	\includegraphics[keepaspectratio=true, width=\textwidth, height=0.23\textheight]{result/img/all/results_baakman_3_120000_sambe_silverman}
	\caption{Set \baakmanThree, \sambe}
	\label{fig:4:results:sambe:baakman3}
\end{subfigure}	
	\caption{Comparative plots for dataset \ferdosiTwoNum, \ferdosiThreeNum, \baakmanTwoNum, and \baakmanThreeNum.}
	\label{fig:4:resuts:multiSphere}
\end{figure*}

\todo[inline]{Report on \cref{fig:4:results:mbe:ferdosi2} and \cref{fig:4:results:mbe:baakman2}.} 
\todo[inline]{Report on \cref{fig:4:results:mbe:ferdosi3} and \cref{fig:4:results:mbe:baakman3}.}

\todo[inline]{General observation of multi sphere datasets.}
	\caption{Comparative plots for dataset \ferdosiTwo, \ferdosiThree, \baakmanTwo, \baakmanThree.}
	\label{fig:4:resuts:multiSphere}
\end{figure*}

\todo[inline]{What do we observe for dataset ferdosi 2, baakman 2, ferdosi 3, baakman 3}
