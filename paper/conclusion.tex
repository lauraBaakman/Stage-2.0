%!TEX root = paper.tex

We have found that the shape-adaptive Modified Breiman Estimator gives results comparable to those of the symmetric Modified Breiman Estimator. Especially near the boundary of the datasets anisotropic kernels prove advantageous. Reviewing the raw data shows that using anisotropic kernel is definitely advantageous. However the number of points where the kernel is too isotropic, or too anisotropic for its local neighborhood negate this advantage. 
% High anisotropy of kernels in noise
One of the main problems is that fine structures in the background result in kernels with a relatively high anisotropy. 
% High anistropy with high density
Whereas Gaussian components are surrounded by points with highly anisotropic kernels, the shape of components with a lower density is hardly detected. Resulting in too isotropic kernels. 

% Further research
These issues might be addressed by increasing the size of the local neighborhood or by deciding the strength. Another possible improvement could be to decide the strength of the anisotropy of the kernel based on the local neighborhood of the associated data point.

% Conclusion
In conclusion shape-adaptive kernels are a promising idea that definitely warrants the research needed to work out the kinks identified in this paper.