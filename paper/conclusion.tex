%!TEX root = paper.tex

\todo[inline]{Review hypothesis!}
\todo[inline]{Introduce seperate section for further research}
\todo[inline]{Summarize the conclusions.}

In conclusion we have found that the shape-adaptive Modified Breiman Estimator gives results comparable to those of the symmetric Modified Breiman Estimator. 
% Boundary Effect
The former is better in estimating the density of points near the boundary of the dataset, especially if the dataset has multiple Gaussian components. Further research is required to determine if this boundary effect also occurs if the data points near th boundaries are not sampled from a uniform random distribution. 
% Does not work for too dense Gaussian components
Furthermore shape-adaptive kernels are outperformed by symmetric kernels in areas where the physical density of points is high. To counteract this one could consider making the value of \KNNK adaptive. A quick survey showed that increasing \KNNK with a factor ten significantly improved the performance of \sambe on some datasets, however it also caused a drop in performance for the remaining dataset. To counteract this one could consider a \KNNK that is adaptive based on the physical density of its neighborhood. 