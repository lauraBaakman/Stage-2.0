%!TEX root = paper.tex

In conclusion we have that the shape adaptive Modified Breiman Estimator gives results comparable to those of the symmetric Modified Breiman Estimator. The estimator with kernels that are responsive to the data is better in estimating the density of points near the boundary of the dataset, especially if the dataset has multiple Gaussian components. Further research is required to determine if this boundary effect also occurs if the data points near th boundaries are not sampled from a uniform random distribution. 

% \todo[inline]{Determine the optimal value of k.}
% \todo[inline]{Further research: If the k-nearest neighbor is more than some distance x away from xi the identity matrix, or better, move between the fixed-width kernel and the shape-adaptive kernel based on the distance of the k-nearest neighbor.}