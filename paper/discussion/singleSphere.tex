%!TEX root = ../paper.tex

%General Discussion
	%Small difference between two estimators
	The most striking observation of \cref{s:results:singleGaussian} is the small difference between the densities approximated by the two estimators. 

		%Explanation: difference are hidden by mean and occlusion in plots: 
		%	Review by plotting SAMBE as a funtion of MBE.
		%Explanation: the kernels are not really shape adaptive
		%	Review by looking at eigenvectors and eigenvalues of the kernels.

		% Ferdosi 1
			% Plot
			% Nearly all points lie on the diagonal -> hardly any differences between points.

			% EigenValues
			% Reviewing the eigenvaleus we find harldy any differences, largest difference: pattern #268 -> all kernels are near spherical. 

		% Baakman 1
			% Plot
			% Nearly all points lie on the diagonal -> hardly any differences between points.

			% EigenValues
			% Eigenvalues indicate five points have kernels with an extreme shape: two long axes and one very short axis. These points are sampled from the noise, have no pattern in localtion. 

		% Baakman 4
			% Plot

			% EigenValues

		% Baakman 5
			% Plot

			% EigenValues

% Dataset Specific Questions
	%Ferdosi 1
	The points that have the largest difference in estimated density in dataset \ferdosiOne all lie near the mean of the Gaussian distribution. Compared to both the true density and the density estimated by the symmetric kernel the shape adaptive kernel severely underestimates the density. Comparing the number of patterns used to estimate the density the densities of these points reveals that \mbe uses approximately twice as many patterns as the shape-adaptive estimator. Which suggests that the domain of the kernels of the nearby points is not large enough. 
	\todo[inline]{How to solve the problem of the to small kernels?}

	%Baakman 1
	% Same point as Ferdosi 1: points with largest differnce lie near the mean of the Gausian, SAMBE uses too few patterns compared to MBE, and severly underestimates the density. 
