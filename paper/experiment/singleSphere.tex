%!TEX root = ../paper.tex

\begin{figure*}
	\centering
	\beforeFinalVersion{Remove ticks and labels.}
	\beforeFinalVersion{Fix the length of the axis of dataset \baakmanTwoNum.}
	%!TEX root = ../paper.tex
%Ferdosi Sets 1
\begin{subfigure}{0.23\textwidth}
	\centering
	\includegraphics[width=\textwidth]{experiment/img/datasetplot_ferdosi_1_60000}
	\caption{Set \ferdosiOne}
	\label{fig:experiment:singlesphere:ferdosi1}
\end{subfigure}
% Baakman 1	
\begin{subfigure}{0.23\textwidth}
	\centering
	\includegraphics[width=\textwidth]{experiment/img/datasetplot_baakman_1_60000}
	\caption{Set \baakmanOne}
	\label{fig:experiment:singlesphere:baakman1}
\end{subfigure}
% Baakman 4
\begin{subfigure}{0.23\textwidth}
	\centering
	\includegraphics[width=\textwidth]{experiment/img/datasetplot_baakman_4_60000}
	\caption{Set \baakmanFour}
	\label{fig:experiment:singlesphere:baakman4}
\end{subfigure}	
% Baakman 5
\begin{subfigure}{0.23\textwidth}
	\centering
	\includegraphics[width=\textwidth]{experiment/img/datasetplot_baakman_5_60000}
	\caption{Set \baakmanFive}
	\label{fig:experiment:singlesphere:baakman5}
\end{subfigure}	
	\caption{Scatter plot representation of the datasets defined in \cref{tab:experiment:singlesphere:sets}. The colors of the different components correspond to the colors used in \cref{tab:experiment:singlesphere:sets}.}
	\label{fig:experiment:singlesphere:sets}
\end{figure*}

\begin{table*}
	\centering
	%!TEX root = ../paper.tex

%!TEX root = ../paper.tex
\sisetup{
	table-format=5.0,
	scientific-notation=false,
	round-mode=places,
	round-precision=1,
	table-number-alignment=center
}
\begin{tabular}{@{}cclSl@{}}
\toprule
Set 			&~						& Component					& {Number} 	& Distribution\\
\midrule
% Ferdosi 1
\ferdosiOne 	&\legendComponentOne	& Trivariate Gaussian 		& 40000		& $\gaussDist{[50, 50, 50]}{\diag(11)}$\\
~ 				&\legendComponentNoise	& Uniform random background	& 20000		& $\uniformDist{[0, 0, 0]}{[100, 100, 100]}$\\
% Baakman 1
\hline
\baakmanOne		&\legendComponentOne	& Trivariate Gaussian 		& 40000		& $\gaussDist{[50, 50, 50]}{\diag([11, \sqrt{11}, \sqrt{11}])}$\\
~ 				&\legendComponentNoise	& Uniform random background	& 20000		& $\uniformDist{[0, 0, 0]}{[100, 100, 100]}$\\
% Baakman 4
\hline
\baakmanFour	&\legendComponentOne	& Trivariate Gaussian 		& 40000		& $\gaussDist{[50, 50, 50]}{\diag([11, 2 * \sqrt{11}, \rfrac{1}{2} \sqrt{11}])}$\\
~ 				&\legendComponentNoise	& Uniform random background	& 20000		& $\uniformDist{[0, 0, 0]}{[100, 100, 100]}$\\
% Baakman 5
\hline
\baakmanFive	&\legendComponentOne	& Trivariate Gaussian 		& 40000		& $\gaussDist{[50, 50, 50]}{\diag([11^2, 11, 1])}$\\
~ 				&\legendComponentNoise	& Uniform random background	& 20000		& $\uniformDist{[0, 0, 0]}{[100, 100, 100]}$\\
\bottomrule
\end{tabular}
	\caption{The containing a single Gaussian distribution next to uniform noise. The column `Number' indicates for each component of the dataset how many data points are sampled from that component. \gaussDist{\varMean}{\varCovarianceMatrix} denotes a Gaussian distribution with mean \varMean and covariance matrix \varCovarianceMatrix. A diagonal matrix with the values $x_1,\, \cdots,\, x_\varDim$ on the diagonal is represented as $\diag([x_1,\,\cdots,\,x_\varDim]])$, a scalar matrix with $x$ on the diagonal is shown as $\diag(x)$. \uniformDist{a}{b} denotes a uniform distribution with its minimum and maximum set to $a$ and $b$, respectively. The colors shown in the second column correspond with the colors used for these components of the data set throughout the paper.} 	
	\label{tab:experiment:singlesphere:sets}
\end{table*}

\todo[inline]{Image of single sphere sets}
\todo[inline]{Discuss \ferdosiOneNum}
\todo[inline]{Discuss \baakmanOneNum}
\todo[inline]{Discuss \baakmanFourNum}
\todo[inline]{Discuss \baakmanFiveNum}

\todo[inline]{Hypothezize on results in general}
\todo[inline]{Hypothezize more specific}