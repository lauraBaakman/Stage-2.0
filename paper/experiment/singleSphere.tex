%!TEX root = ../paper.tex

\begin{figure*}
	\centering
	\beforeFinalVersion{Remove ticks and labels.}
	%!TEX root = ../paper.tex
%Ferdosi Sets 1
\begin{subfigure}{0.23\textwidth}
	\centering
	\includegraphics[width=\textwidth]{experiment/img/datasetplot_ferdosi_1_60000}
	\caption{Set \ferdosiOne}
	\label{fig:experiment:singlesphere:ferdosi1}
\end{subfigure}
% Baakman 1	
\begin{subfigure}{0.23\textwidth}
	\centering
	\includegraphics[width=\textwidth]{experiment/img/datasetplot_baakman_1_60000}
	\caption{Set \baakmanOne}
	\label{fig:experiment:singlesphere:baakman1}
\end{subfigure}
% Baakman 4
\begin{subfigure}{0.23\textwidth}
	\centering
	\includegraphics[width=\textwidth]{experiment/img/datasetplot_baakman_4_60000}
	\caption{Set \baakmanFour}
	\label{fig:experiment:singlesphere:baakman4}
\end{subfigure}	
% Baakman 5
\begin{subfigure}{0.23\textwidth}
	\centering
	\includegraphics[width=\textwidth]{experiment/img/datasetplot_baakman_5_60000}
	\caption{Set \baakmanFive}
	\label{fig:experiment:singlesphere:baakman5}
\end{subfigure}	
	\caption{Scatter plot representation of the datasets defined in \cref{tab:experiment:singlesphere:sets}. The colors of the different components correspond to the colors used in \cref{tab:experiment:singlesphere:sets}.}
	\label{fig:experiment:singlesphere:sets}
\end{figure*}

\begin{table*}
	\centering
	%!TEX root = ../paper.tex

%!TEX root = ../paper.tex
\sisetup{
	table-format=5.0,
	scientific-notation=false,
	round-mode=places,
	round-precision=1,
	table-number-alignment=center
}
\begin{tabular}{@{}cclSl@{}}
\toprule
Set 			&~						& Component					& {Number} 	& Distribution\\
\midrule
% Ferdosi 1
\ferdosiOne 	&\legendComponentOne	& Trivariate Gaussian 		& 40000		& $\gaussDist{[50, 50, 50]}{\diag(11)}$\\
~ 				&\legendComponentNoise	& Uniform random background	& 20000		& $\uniformDist{[0, 0, 0]}{[100, 100, 100]}$\\
% Baakman 1
\hline
\baakmanOne		&\legendComponentOne	& Trivariate Gaussian 		& 40000		& $\gaussDist{[50, 50, 50]}{\diag([11, \sqrt{11}, \sqrt{11}])}$\\
~ 				&\legendComponentNoise	& Uniform random background	& 20000		& $\uniformDist{[0, 0, 0]}{[100, 100, 100]}$\\
% Baakman 4
\hline
\baakmanFour	&\legendComponentOne	& Trivariate Gaussian 		& 40000		& $\gaussDist{[50, 50, 50]}{\diag([11, 2 * \sqrt{11}, \rfrac{1}{2} \sqrt{11}])}$\\
~ 				&\legendComponentNoise	& Uniform random background	& 20000		& $\uniformDist{[0, 0, 0]}{[100, 100, 100]}$\\
% Baakman 5
\hline
\baakmanFive	&\legendComponentOne	& Trivariate Gaussian 		& 40000		& $\gaussDist{[50, 50, 50]}{\diag([11^2, 11, 1])}$\\
~ 				&\legendComponentNoise	& Uniform random background	& 20000		& $\uniformDist{[0, 0, 0]}{[100, 100, 100]}$\\
\bottomrule
\end{tabular}
	\caption{The datasets containing a single Gaussian distribution next to uniform noise. The column `Number' indicates for each component of the dataset how many data points are sampled from that component. \gaussDist{\varMean}{\varCovarianceMatrix} denotes a Gaussian distribution with mean \varMean and covariance matrix \varCovarianceMatrix. A diagonal matrix with the values $x_1,\, \cdots,\, x_\varDim$ on the diagonal is represented as $\diag([x_1,\,\cdots,\,x_\varDim]])$, a scalar matrix with $x$ on the diagonal is shown as $\diag(x)$. \uniformDist{a}{b} denotes a uniform distribution with its minimum and maximum set to $a$ and $b$, respectively. The colors shown in the second column correspond with the colors used for these components of the data set throughout the paper.} 	
	\label{tab:experiment:singlesphere:sets}
\end{table*}

\Cref{fig:experiment:singlesphere:sets} shows a scatter plot representation of the datasets containing a single Gaussian distribution defined in \cref{tab:experiment:singlesphere:sets}. 

% General
The Gaussian components of these datasets progress from a sphere, \ie dataset \ferdosiOne, to an increasingly more elongated ellipsoid. This makes it possible to investigate the influence of how strongly elongated the distribution is on the density estimate. 
	% Ferdosi One
	The first dataset is a simple spherical Gaussian distribution centered in a uniform random background. 
	% Baakman One
	Dataset \baakmanOne is created from \ferdosiOne by squaring one of the eigenvalues of the covariance matrix, and taking the square root of the other two eigenvalues, without changing the eigenvectors. The resulting covariance matrix defines an eigenellipse with the same volume as the one defined by \ferdosiOne.
	% Baakman Four
	The Gaussian component of dataset \baakmanFour elongates the eigenellipse of the Gaussian component by lengthening one of the minor axes, and shortening the other.
	% Baakman Five
	Dataset \baakmanFive has a Gaussian component that is spread out more along the y-axis and less along the z-axis than that component in dataset \baakmanFour. 

% Hypothesis
	% Ferdosi 1
	We expect the Modified Breiman Estimator and its shape-adaptive cousin to perform comparably on dataset \ferdosiOne, since due to the symmetric shape of the Gaussian distribution no advantage should be gained by using a shape-adaptive kernel. 
	% Baakman 1, 4, 5
	As the Gaussian distribution is more and more elongated, the advantage of using \sambe should become more pronounced. 